
% Default to the notebook output style

    


% Inherit from the specified cell style.




    
\documentclass[11pt]{article}

    
    
    \usepackage[T1]{fontenc}
    % Nicer default font (+ math font) than Computer Modern for most use cases
    \usepackage{mathpazo}

    % Basic figure setup, for now with no caption control since it's done
    % automatically by Pandoc (which extracts ![](path) syntax from Markdown).
    \usepackage{graphicx}
    % We will generate all images so they have a width \maxwidth. This means
    % that they will get their normal width if they fit onto the page, but
    % are scaled down if they would overflow the margins.
    \makeatletter
    \def\maxwidth{\ifdim\Gin@nat@width>\linewidth\linewidth
    \else\Gin@nat@width\fi}
    \makeatother
    \let\Oldincludegraphics\includegraphics
    % Set max figure width to be 80% of text width, for now hardcoded.
    \renewcommand{\includegraphics}[1]{\Oldincludegraphics[width=.8\maxwidth]{#1}}
    % Ensure that by default, figures have no caption (until we provide a
    % proper Figure object with a Caption API and a way to capture that
    % in the conversion process - todo).
    \usepackage{caption}
    \DeclareCaptionLabelFormat{nolabel}{}
    \captionsetup{labelformat=nolabel}

    \usepackage{adjustbox} % Used to constrain images to a maximum size 
    \usepackage{xcolor} % Allow colors to be defined
    \usepackage{enumerate} % Needed for markdown enumerations to work
    \usepackage{geometry} % Used to adjust the document margins
    \usepackage{amsmath} % Equations
    \usepackage{amssymb} % Equations
    \usepackage{textcomp} % defines textquotesingle
    % Hack from http://tex.stackexchange.com/a/47451/13684:
    \AtBeginDocument{%
        \def\PYZsq{\textquotesingle}% Upright quotes in Pygmentized code
    }
    \usepackage{upquote} % Upright quotes for verbatim code
    \usepackage{eurosym} % defines \euro
    \usepackage[mathletters]{ucs} % Extended unicode (utf-8) support
    \usepackage[utf8x]{inputenc} % Allow utf-8 characters in the tex document
    \usepackage{fancyvrb} % verbatim replacement that allows latex
    \usepackage{grffile} % extends the file name processing of package graphics 
                         % to support a larger range 
    % The hyperref package gives us a pdf with properly built
    % internal navigation ('pdf bookmarks' for the table of contents,
    % internal cross-reference links, web links for URLs, etc.)
    \usepackage{hyperref}
    \usepackage{longtable} % longtable support required by pandoc >1.10
    \usepackage{booktabs}  % table support for pandoc > 1.12.2
    \usepackage[inline]{enumitem} % IRkernel/repr support (it uses the enumerate* environment)
    \usepackage[normalem]{ulem} % ulem is needed to support strikethroughs (\sout)
                                % normalem makes italics be italics, not underlines
    

    
    
    % Colors for the hyperref package
    \definecolor{urlcolor}{rgb}{0,.145,.698}
    \definecolor{linkcolor}{rgb}{.71,0.21,0.01}
    \definecolor{citecolor}{rgb}{.12,.54,.11}

    % ANSI colors
    \definecolor{ansi-black}{HTML}{3E424D}
    \definecolor{ansi-black-intense}{HTML}{282C36}
    \definecolor{ansi-red}{HTML}{E75C58}
    \definecolor{ansi-red-intense}{HTML}{B22B31}
    \definecolor{ansi-green}{HTML}{00A250}
    \definecolor{ansi-green-intense}{HTML}{007427}
    \definecolor{ansi-yellow}{HTML}{DDB62B}
    \definecolor{ansi-yellow-intense}{HTML}{B27D12}
    \definecolor{ansi-blue}{HTML}{208FFB}
    \definecolor{ansi-blue-intense}{HTML}{0065CA}
    \definecolor{ansi-magenta}{HTML}{D160C4}
    \definecolor{ansi-magenta-intense}{HTML}{A03196}
    \definecolor{ansi-cyan}{HTML}{60C6C8}
    \definecolor{ansi-cyan-intense}{HTML}{258F8F}
    \definecolor{ansi-white}{HTML}{C5C1B4}
    \definecolor{ansi-white-intense}{HTML}{A1A6B2}

    % commands and environments needed by pandoc snippets
    % extracted from the output of `pandoc -s`
    \providecommand{\tightlist}{%
      \setlength{\itemsep}{0pt}\setlength{\parskip}{0pt}}
    \DefineVerbatimEnvironment{Highlighting}{Verbatim}{commandchars=\\\{\}}
    % Add ',fontsize=\small' for more characters per line
    \newenvironment{Shaded}{}{}
    \newcommand{\KeywordTok}[1]{\textcolor[rgb]{0.00,0.44,0.13}{\textbf{{#1}}}}
    \newcommand{\DataTypeTok}[1]{\textcolor[rgb]{0.56,0.13,0.00}{{#1}}}
    \newcommand{\DecValTok}[1]{\textcolor[rgb]{0.25,0.63,0.44}{{#1}}}
    \newcommand{\BaseNTok}[1]{\textcolor[rgb]{0.25,0.63,0.44}{{#1}}}
    \newcommand{\FloatTok}[1]{\textcolor[rgb]{0.25,0.63,0.44}{{#1}}}
    \newcommand{\CharTok}[1]{\textcolor[rgb]{0.25,0.44,0.63}{{#1}}}
    \newcommand{\StringTok}[1]{\textcolor[rgb]{0.25,0.44,0.63}{{#1}}}
    \newcommand{\CommentTok}[1]{\textcolor[rgb]{0.38,0.63,0.69}{\textit{{#1}}}}
    \newcommand{\OtherTok}[1]{\textcolor[rgb]{0.00,0.44,0.13}{{#1}}}
    \newcommand{\AlertTok}[1]{\textcolor[rgb]{1.00,0.00,0.00}{\textbf{{#1}}}}
    \newcommand{\FunctionTok}[1]{\textcolor[rgb]{0.02,0.16,0.49}{{#1}}}
    \newcommand{\RegionMarkerTok}[1]{{#1}}
    \newcommand{\ErrorTok}[1]{\textcolor[rgb]{1.00,0.00,0.00}{\textbf{{#1}}}}
    \newcommand{\NormalTok}[1]{{#1}}
    
    % Additional commands for more recent versions of Pandoc
    \newcommand{\ConstantTok}[1]{\textcolor[rgb]{0.53,0.00,0.00}{{#1}}}
    \newcommand{\SpecialCharTok}[1]{\textcolor[rgb]{0.25,0.44,0.63}{{#1}}}
    \newcommand{\VerbatimStringTok}[1]{\textcolor[rgb]{0.25,0.44,0.63}{{#1}}}
    \newcommand{\SpecialStringTok}[1]{\textcolor[rgb]{0.73,0.40,0.53}{{#1}}}
    \newcommand{\ImportTok}[1]{{#1}}
    \newcommand{\DocumentationTok}[1]{\textcolor[rgb]{0.73,0.13,0.13}{\textit{{#1}}}}
    \newcommand{\AnnotationTok}[1]{\textcolor[rgb]{0.38,0.63,0.69}{\textbf{\textit{{#1}}}}}
    \newcommand{\CommentVarTok}[1]{\textcolor[rgb]{0.38,0.63,0.69}{\textbf{\textit{{#1}}}}}
    \newcommand{\VariableTok}[1]{\textcolor[rgb]{0.10,0.09,0.49}{{#1}}}
    \newcommand{\ControlFlowTok}[1]{\textcolor[rgb]{0.00,0.44,0.13}{\textbf{{#1}}}}
    \newcommand{\OperatorTok}[1]{\textcolor[rgb]{0.40,0.40,0.40}{{#1}}}
    \newcommand{\BuiltInTok}[1]{{#1}}
    \newcommand{\ExtensionTok}[1]{{#1}}
    \newcommand{\PreprocessorTok}[1]{\textcolor[rgb]{0.74,0.48,0.00}{{#1}}}
    \newcommand{\AttributeTok}[1]{\textcolor[rgb]{0.49,0.56,0.16}{{#1}}}
    \newcommand{\InformationTok}[1]{\textcolor[rgb]{0.38,0.63,0.69}{\textbf{\textit{{#1}}}}}
    \newcommand{\WarningTok}[1]{\textcolor[rgb]{0.38,0.63,0.69}{\textbf{\textit{{#1}}}}}
    
    
    % Define a nice break command that doesn't care if a line doesn't already
    % exist.
    \def\br{\hspace*{\fill} \\* }
    % Math Jax compatability definitions
    \def\gt{>}
    \def\lt{<}
    % Document parameters
    \title{MinimalProject}
    
    
    

    % Pygments definitions
    
\makeatletter
\def\PY@reset{\let\PY@it=\relax \let\PY@bf=\relax%
    \let\PY@ul=\relax \let\PY@tc=\relax%
    \let\PY@bc=\relax \let\PY@ff=\relax}
\def\PY@tok#1{\csname PY@tok@#1\endcsname}
\def\PY@toks#1+{\ifx\relax#1\empty\else%
    \PY@tok{#1}\expandafter\PY@toks\fi}
\def\PY@do#1{\PY@bc{\PY@tc{\PY@ul{%
    \PY@it{\PY@bf{\PY@ff{#1}}}}}}}
\def\PY#1#2{\PY@reset\PY@toks#1+\relax+\PY@do{#2}}

\expandafter\def\csname PY@tok@w\endcsname{\def\PY@tc##1{\textcolor[rgb]{0.73,0.73,0.73}{##1}}}
\expandafter\def\csname PY@tok@c\endcsname{\let\PY@it=\textit\def\PY@tc##1{\textcolor[rgb]{0.25,0.50,0.50}{##1}}}
\expandafter\def\csname PY@tok@cp\endcsname{\def\PY@tc##1{\textcolor[rgb]{0.74,0.48,0.00}{##1}}}
\expandafter\def\csname PY@tok@k\endcsname{\let\PY@bf=\textbf\def\PY@tc##1{\textcolor[rgb]{0.00,0.50,0.00}{##1}}}
\expandafter\def\csname PY@tok@kp\endcsname{\def\PY@tc##1{\textcolor[rgb]{0.00,0.50,0.00}{##1}}}
\expandafter\def\csname PY@tok@kt\endcsname{\def\PY@tc##1{\textcolor[rgb]{0.69,0.00,0.25}{##1}}}
\expandafter\def\csname PY@tok@o\endcsname{\def\PY@tc##1{\textcolor[rgb]{0.40,0.40,0.40}{##1}}}
\expandafter\def\csname PY@tok@ow\endcsname{\let\PY@bf=\textbf\def\PY@tc##1{\textcolor[rgb]{0.67,0.13,1.00}{##1}}}
\expandafter\def\csname PY@tok@nb\endcsname{\def\PY@tc##1{\textcolor[rgb]{0.00,0.50,0.00}{##1}}}
\expandafter\def\csname PY@tok@nf\endcsname{\def\PY@tc##1{\textcolor[rgb]{0.00,0.00,1.00}{##1}}}
\expandafter\def\csname PY@tok@nc\endcsname{\let\PY@bf=\textbf\def\PY@tc##1{\textcolor[rgb]{0.00,0.00,1.00}{##1}}}
\expandafter\def\csname PY@tok@nn\endcsname{\let\PY@bf=\textbf\def\PY@tc##1{\textcolor[rgb]{0.00,0.00,1.00}{##1}}}
\expandafter\def\csname PY@tok@ne\endcsname{\let\PY@bf=\textbf\def\PY@tc##1{\textcolor[rgb]{0.82,0.25,0.23}{##1}}}
\expandafter\def\csname PY@tok@nv\endcsname{\def\PY@tc##1{\textcolor[rgb]{0.10,0.09,0.49}{##1}}}
\expandafter\def\csname PY@tok@no\endcsname{\def\PY@tc##1{\textcolor[rgb]{0.53,0.00,0.00}{##1}}}
\expandafter\def\csname PY@tok@nl\endcsname{\def\PY@tc##1{\textcolor[rgb]{0.63,0.63,0.00}{##1}}}
\expandafter\def\csname PY@tok@ni\endcsname{\let\PY@bf=\textbf\def\PY@tc##1{\textcolor[rgb]{0.60,0.60,0.60}{##1}}}
\expandafter\def\csname PY@tok@na\endcsname{\def\PY@tc##1{\textcolor[rgb]{0.49,0.56,0.16}{##1}}}
\expandafter\def\csname PY@tok@nt\endcsname{\let\PY@bf=\textbf\def\PY@tc##1{\textcolor[rgb]{0.00,0.50,0.00}{##1}}}
\expandafter\def\csname PY@tok@nd\endcsname{\def\PY@tc##1{\textcolor[rgb]{0.67,0.13,1.00}{##1}}}
\expandafter\def\csname PY@tok@s\endcsname{\def\PY@tc##1{\textcolor[rgb]{0.73,0.13,0.13}{##1}}}
\expandafter\def\csname PY@tok@sd\endcsname{\let\PY@it=\textit\def\PY@tc##1{\textcolor[rgb]{0.73,0.13,0.13}{##1}}}
\expandafter\def\csname PY@tok@si\endcsname{\let\PY@bf=\textbf\def\PY@tc##1{\textcolor[rgb]{0.73,0.40,0.53}{##1}}}
\expandafter\def\csname PY@tok@se\endcsname{\let\PY@bf=\textbf\def\PY@tc##1{\textcolor[rgb]{0.73,0.40,0.13}{##1}}}
\expandafter\def\csname PY@tok@sr\endcsname{\def\PY@tc##1{\textcolor[rgb]{0.73,0.40,0.53}{##1}}}
\expandafter\def\csname PY@tok@ss\endcsname{\def\PY@tc##1{\textcolor[rgb]{0.10,0.09,0.49}{##1}}}
\expandafter\def\csname PY@tok@sx\endcsname{\def\PY@tc##1{\textcolor[rgb]{0.00,0.50,0.00}{##1}}}
\expandafter\def\csname PY@tok@m\endcsname{\def\PY@tc##1{\textcolor[rgb]{0.40,0.40,0.40}{##1}}}
\expandafter\def\csname PY@tok@gh\endcsname{\let\PY@bf=\textbf\def\PY@tc##1{\textcolor[rgb]{0.00,0.00,0.50}{##1}}}
\expandafter\def\csname PY@tok@gu\endcsname{\let\PY@bf=\textbf\def\PY@tc##1{\textcolor[rgb]{0.50,0.00,0.50}{##1}}}
\expandafter\def\csname PY@tok@gd\endcsname{\def\PY@tc##1{\textcolor[rgb]{0.63,0.00,0.00}{##1}}}
\expandafter\def\csname PY@tok@gi\endcsname{\def\PY@tc##1{\textcolor[rgb]{0.00,0.63,0.00}{##1}}}
\expandafter\def\csname PY@tok@gr\endcsname{\def\PY@tc##1{\textcolor[rgb]{1.00,0.00,0.00}{##1}}}
\expandafter\def\csname PY@tok@ge\endcsname{\let\PY@it=\textit}
\expandafter\def\csname PY@tok@gs\endcsname{\let\PY@bf=\textbf}
\expandafter\def\csname PY@tok@gp\endcsname{\let\PY@bf=\textbf\def\PY@tc##1{\textcolor[rgb]{0.00,0.00,0.50}{##1}}}
\expandafter\def\csname PY@tok@go\endcsname{\def\PY@tc##1{\textcolor[rgb]{0.53,0.53,0.53}{##1}}}
\expandafter\def\csname PY@tok@gt\endcsname{\def\PY@tc##1{\textcolor[rgb]{0.00,0.27,0.87}{##1}}}
\expandafter\def\csname PY@tok@err\endcsname{\def\PY@bc##1{\setlength{\fboxsep}{0pt}\fcolorbox[rgb]{1.00,0.00,0.00}{1,1,1}{\strut ##1}}}
\expandafter\def\csname PY@tok@kc\endcsname{\let\PY@bf=\textbf\def\PY@tc##1{\textcolor[rgb]{0.00,0.50,0.00}{##1}}}
\expandafter\def\csname PY@tok@kd\endcsname{\let\PY@bf=\textbf\def\PY@tc##1{\textcolor[rgb]{0.00,0.50,0.00}{##1}}}
\expandafter\def\csname PY@tok@kn\endcsname{\let\PY@bf=\textbf\def\PY@tc##1{\textcolor[rgb]{0.00,0.50,0.00}{##1}}}
\expandafter\def\csname PY@tok@kr\endcsname{\let\PY@bf=\textbf\def\PY@tc##1{\textcolor[rgb]{0.00,0.50,0.00}{##1}}}
\expandafter\def\csname PY@tok@bp\endcsname{\def\PY@tc##1{\textcolor[rgb]{0.00,0.50,0.00}{##1}}}
\expandafter\def\csname PY@tok@fm\endcsname{\def\PY@tc##1{\textcolor[rgb]{0.00,0.00,1.00}{##1}}}
\expandafter\def\csname PY@tok@vc\endcsname{\def\PY@tc##1{\textcolor[rgb]{0.10,0.09,0.49}{##1}}}
\expandafter\def\csname PY@tok@vg\endcsname{\def\PY@tc##1{\textcolor[rgb]{0.10,0.09,0.49}{##1}}}
\expandafter\def\csname PY@tok@vi\endcsname{\def\PY@tc##1{\textcolor[rgb]{0.10,0.09,0.49}{##1}}}
\expandafter\def\csname PY@tok@vm\endcsname{\def\PY@tc##1{\textcolor[rgb]{0.10,0.09,0.49}{##1}}}
\expandafter\def\csname PY@tok@sa\endcsname{\def\PY@tc##1{\textcolor[rgb]{0.73,0.13,0.13}{##1}}}
\expandafter\def\csname PY@tok@sb\endcsname{\def\PY@tc##1{\textcolor[rgb]{0.73,0.13,0.13}{##1}}}
\expandafter\def\csname PY@tok@sc\endcsname{\def\PY@tc##1{\textcolor[rgb]{0.73,0.13,0.13}{##1}}}
\expandafter\def\csname PY@tok@dl\endcsname{\def\PY@tc##1{\textcolor[rgb]{0.73,0.13,0.13}{##1}}}
\expandafter\def\csname PY@tok@s2\endcsname{\def\PY@tc##1{\textcolor[rgb]{0.73,0.13,0.13}{##1}}}
\expandafter\def\csname PY@tok@sh\endcsname{\def\PY@tc##1{\textcolor[rgb]{0.73,0.13,0.13}{##1}}}
\expandafter\def\csname PY@tok@s1\endcsname{\def\PY@tc##1{\textcolor[rgb]{0.73,0.13,0.13}{##1}}}
\expandafter\def\csname PY@tok@mb\endcsname{\def\PY@tc##1{\textcolor[rgb]{0.40,0.40,0.40}{##1}}}
\expandafter\def\csname PY@tok@mf\endcsname{\def\PY@tc##1{\textcolor[rgb]{0.40,0.40,0.40}{##1}}}
\expandafter\def\csname PY@tok@mh\endcsname{\def\PY@tc##1{\textcolor[rgb]{0.40,0.40,0.40}{##1}}}
\expandafter\def\csname PY@tok@mi\endcsname{\def\PY@tc##1{\textcolor[rgb]{0.40,0.40,0.40}{##1}}}
\expandafter\def\csname PY@tok@il\endcsname{\def\PY@tc##1{\textcolor[rgb]{0.40,0.40,0.40}{##1}}}
\expandafter\def\csname PY@tok@mo\endcsname{\def\PY@tc##1{\textcolor[rgb]{0.40,0.40,0.40}{##1}}}
\expandafter\def\csname PY@tok@ch\endcsname{\let\PY@it=\textit\def\PY@tc##1{\textcolor[rgb]{0.25,0.50,0.50}{##1}}}
\expandafter\def\csname PY@tok@cm\endcsname{\let\PY@it=\textit\def\PY@tc##1{\textcolor[rgb]{0.25,0.50,0.50}{##1}}}
\expandafter\def\csname PY@tok@cpf\endcsname{\let\PY@it=\textit\def\PY@tc##1{\textcolor[rgb]{0.25,0.50,0.50}{##1}}}
\expandafter\def\csname PY@tok@c1\endcsname{\let\PY@it=\textit\def\PY@tc##1{\textcolor[rgb]{0.25,0.50,0.50}{##1}}}
\expandafter\def\csname PY@tok@cs\endcsname{\let\PY@it=\textit\def\PY@tc##1{\textcolor[rgb]{0.25,0.50,0.50}{##1}}}

\def\PYZbs{\char`\\}
\def\PYZus{\char`\_}
\def\PYZob{\char`\{}
\def\PYZcb{\char`\}}
\def\PYZca{\char`\^}
\def\PYZam{\char`\&}
\def\PYZlt{\char`\<}
\def\PYZgt{\char`\>}
\def\PYZsh{\char`\#}
\def\PYZpc{\char`\%}
\def\PYZdl{\char`\$}
\def\PYZhy{\char`\-}
\def\PYZsq{\char`\'}
\def\PYZdq{\char`\"}
\def\PYZti{\char`\~}
% for compatibility with earlier versions
\def\PYZat{@}
\def\PYZlb{[}
\def\PYZrb{]}
\makeatother


    % Exact colors from NB
    \definecolor{incolor}{rgb}{0.0, 0.0, 0.5}
    \definecolor{outcolor}{rgb}{0.545, 0.0, 0.0}



    
    % Prevent overflowing lines due to hard-to-break entities
    \sloppy 
    % Setup hyperref package
    \hypersetup{
      breaklinks=true,  % so long urls are correctly broken across lines
      colorlinks=true,
      urlcolor=urlcolor,
      linkcolor=linkcolor,
      citecolor=citecolor,
      }
    % Slightly bigger margins than the latex defaults
    
    \geometry{verbose,tmargin=1in,bmargin=1in,lmargin=1in,rmargin=1in}
    
    

    \begin{document}
    
    
    \maketitle
    
    

    
    \section{Final project}\label{final-project}

Kendra Chalkley

CS 559 Machine Learning

June 26, 2018

\subsection{Motivation and Goals}\label{motivation-and-goals}

Goal: Write an agglomerative clustering algorithm for Reddit data

\subsubsection{Collecting this Data}\label{collecting-this-data}

I initially collected this data as part of a replication project
following
\href{http://journals.sagepub.com/doi/abs/10.1177/2167702617747074}{In
and Absolute State} (Al-Mosaiwi and Johnstone, 2018). The study explored
the use of 'asbolutist' words (always, never, completely, etc) in
anxiety and depression related forums and found that absolutist words
were used at significantly higher rate in these groups than in controls.
The paper also compares the frequency of various other collections of
words or dictionaries (pronouns, negative emotion words, et al)
available through a software called LIWC for Linguistic Inquiry and Word
Count.

I wanted to use distributed computing to investigate these frequencies
in a larger body of work, specifically Reddit data. Reddit.com is a
website on which users can create and contribute to forums called
'subreddits.' Subreddits cover a vast number of subjects including
various mental illnesses, computing and gaming, sports, politics, the
economy, and a large number of 'adult' topics. The dataset I've
collected to use here is in csv format including information about
approximately 1200 subreddits to which users contibutes text posts of at
least 100 words in length. It includes the following columns;

\begin{itemize}
\item
  subreddit the name a subreddit (\textless{}=20 characters)
\item
  posts a count of posts with 100 or more words in each subreddit
\item
  wordcount a sum of the total number of words in all posts in that
  subreddit
\item
  65 additional columns, each containing the frequency of words in a
  specific dictionary
\item
  64 from an outdated version of LIWC plus the absolutist dictionary
  mentioned above.
\item
  freq the frequency of words from each dictionary calculated as:
\end{itemize}

\[\frac{number\ of\ words\ matching\ dictionary}{number\ of\ total\ words}\]

\subsubsection{Why agglomerative
clustering?}\label{why-agglomerative-clustering}

Original study which I collected this data to replicate selected the
forums they used by google search. They analyzed a small collection of
forumns on a select number of topics. To approximate this approach, I
tried running K-means clustering on this dataset for K between 5 and 25.
No value of K clustering was obviously more effective than another, and,
when projected onto the first two principal components, cluster
boundaries seemed arbitrary and difficult to interpret.

Hierarchical clustering is my next step in exploring this dataset, as I
hope it will provide more insight into the relationship between
clusters.

\subsection{The Process}\label{the-process}

\subsubsection{Packages and Reading
data}\label{packages-and-reading-data}

For this project I am using numpy to store my purely numerical data
execute operations over large arrays and matrixes. Pandas provides
additional support for numpy arrays and is useful for accessing columns
of data by a variable name and for labeling rows with an index name. I
am also using Matplotlib's pyplot to produce a few scatter plots of the
data projected onto its first two components.

    \begin{Verbatim}[commandchars=\\\{\}]
{\color{incolor}In [{\color{incolor}1}]:} \PY{k+kn}{import} \PY{n+nn}{pandas} \PY{k}{as} \PY{n+nn}{pd}
        \PY{k+kn}{import} \PY{n+nn}{numpy} \PY{k}{as} \PY{n+nn}{np}
        \PY{k+kn}{import} \PY{n+nn}{matplotlib}\PY{n+nn}{.}\PY{n+nn}{pyplot} \PY{k}{as} \PY{n+nn}{plt}
        \PY{o}{\PYZpc{}} \PY{n}{matplotlib} \PY{n}{inline}
        
        
        \PY{n}{inputs}\PY{o}{=} \PY{n}{pd}\PY{o}{.}\PY{n}{read\PYZus{}csv}\PY{p}{(}\PY{l+s+s1}{\PYZsq{}}\PY{l+s+s1}{medoutput.csv}\PY{l+s+s1}{\PYZsq{}}\PY{p}{)}
        \PY{n}{subset\PYZus{}inputs}\PY{o}{=} \PY{n}{pd}\PY{o}{.}\PY{n}{read\PYZus{}csv}\PY{p}{(}\PY{l+s+s1}{\PYZsq{}}\PY{l+s+s1}{subset\PYZus{}medoutput.csv}\PY{l+s+s1}{\PYZsq{}}\PY{p}{)}
        
        \PY{n}{freqs}\PY{o}{=}\PY{n}{inputs}\PY{o}{.}\PY{n}{loc}\PY{p}{[}\PY{p}{:}\PY{p}{,}\PY{n+nb}{list}\PY{p}{(}\PY{n+nb}{set}\PY{p}{(}\PY{n}{inputs}\PY{o}{.}\PY{n}{columns}\PY{p}{)} \PY{o}{\PYZhy{}} \PY{n+nb}{set}\PY{p}{(}\PY{p}{(}\PY{l+s+s1}{\PYZsq{}}\PY{l+s+s1}{subreddit}\PY{l+s+s1}{\PYZsq{}}\PY{p}{,}\PY{l+s+s1}{\PYZsq{}}\PY{l+s+s1}{count(1)}\PY{l+s+s1}{\PYZsq{}}\PY{p}{,}\PY{l+s+s1}{\PYZsq{}}\PY{l+s+s1}{sum(wordcount)}\PY{l+s+s1}{\PYZsq{}}\PY{p}{)}\PY{p}{)}\PY{p}{)}\PY{p}{]}
        \PY{n}{subset\PYZus{}freqs}\PY{o}{=}\PY{n}{subset\PYZus{}inputs}\PY{o}{.}\PY{n}{loc}\PY{p}{[}\PY{p}{:}\PY{p}{,}\PY{n+nb}{list}\PY{p}{(}\PY{n+nb}{set}\PY{p}{(}\PY{n}{inputs}\PY{o}{.}\PY{n}{columns}\PY{p}{)} \PY{o}{\PYZhy{}} \PY{n+nb}{set}\PY{p}{(}\PY{p}{(}\PY{l+s+s1}{\PYZsq{}}\PY{l+s+s1}{subreddit}\PY{l+s+s1}{\PYZsq{}}\PY{p}{,}\PY{l+s+s1}{\PYZsq{}}\PY{l+s+s1}{count(1)}\PY{l+s+s1}{\PYZsq{}}\PY{p}{,}\PY{l+s+s1}{\PYZsq{}}\PY{l+s+s1}{sum(wordcount)}\PY{l+s+s1}{\PYZsq{}}\PY{p}{)}\PY{p}{)}\PY{p}{)}\PY{p}{]}
\end{Verbatim}


    \subsubsection{Preprocessing}\label{preprocessing}

Because statistics for subreddits with a smaller number of posts vary so
much more than subreddits with for which data was calculated from larger
sample size, I've also curated a smaller subset of the data including
only the half of subreddits with the most posts. This I read into to
variable subset\_inputs from a local csv.

Because this is my own dataset, the only additional preprocessing
necessary was to normalize the values. The functions below normalize
each column of the dataset to a range between 0 and 1, so that variation
in dictionaries that have naturally higher values, such as functional
word frequency, would not have greater impact on PCA and clustering than
variation in dictionaries which occur with lower frequency: leisure
words for example.

    \begin{Verbatim}[commandchars=\\\{\}]
{\color{incolor}In [{\color{incolor}2}]:} \PY{k}{def} \PY{n+nf}{normalize}\PY{p}{(}\PY{n}{vector}\PY{p}{)}\PY{p}{:}
            \PY{n}{v2}\PY{o}{=}\PY{n}{vector}
            \PY{n}{vmin}\PY{o}{=}\PY{n+nb}{float}\PY{p}{(}\PY{n+nb}{min}\PY{p}{(}\PY{n}{vector}\PY{p}{)}\PY{p}{)}
            \PY{n}{vmax}\PY{o}{=}\PY{n+nb}{float}\PY{p}{(}\PY{n+nb}{max}\PY{p}{(}\PY{n}{vector}\PY{p}{)}\PY{p}{)}
            \PY{n}{vrange}\PY{o}{=}\PY{n}{vmax}\PY{o}{\PYZhy{}}\PY{n}{vmin}
        
        \PY{c+c1}{\PYZsh{}    print(\PYZsq{}vector\PYZsq{}, vector, vrange)}
            
            \PY{n}{l}\PY{o}{=}\PY{n+nb}{len}\PY{p}{(}\PY{n}{vector}\PY{p}{)}
            \PY{k}{for} \PY{n}{i} \PY{o+ow}{in} \PY{n+nb}{range}\PY{p}{(}\PY{l+m+mi}{0}\PY{p}{,}\PY{n}{l}\PY{p}{)}\PY{p}{:}
                \PY{n}{dif}\PY{o}{=}\PY{n+nb}{float}\PY{p}{(}\PY{n}{vector}\PY{p}{[}\PY{n}{i}\PY{p}{]}\PY{p}{)}\PY{o}{\PYZhy{}}\PY{n}{vmin}
        \PY{c+c1}{\PYZsh{}        print(\PYZsq{}dif\PYZsq{},dif, vrange, dif/vrange)}
                \PY{n}{v2}\PY{p}{[}\PY{n}{i}\PY{p}{]}\PY{o}{=}\PY{n}{dif}\PY{o}{/}\PY{n}{vrange}
        \PY{c+c1}{\PYZsh{}        print(vector[i])}
            \PY{k}{return} \PY{n}{v2}
        
        \PY{k}{def} \PY{n+nf}{normalizeall}\PY{p}{(}\PY{n}{df}\PY{p}{)}\PY{p}{:}
            \PY{n}{df2}\PY{o}{=}\PY{n}{df}
            \PY{n}{i}\PY{o}{=}\PY{n}{df}\PY{o}{.}\PY{n}{shape}\PY{p}{[}\PY{l+m+mi}{1}\PY{p}{]}
            \PY{k}{for} \PY{n}{j} \PY{o+ow}{in} \PY{n+nb}{range}\PY{p}{(}\PY{l+m+mi}{0}\PY{p}{,}\PY{n}{i}\PY{p}{)}\PY{p}{:}
                \PY{n}{df2}\PY{p}{[}\PY{p}{:}\PY{p}{,}\PY{n}{j}\PY{p}{]}\PY{o}{=}\PY{n}{normalize}\PY{p}{(}\PY{n}{df2}\PY{p}{[}\PY{p}{:}\PY{p}{,}\PY{n}{j}\PY{p}{]}\PY{p}{)}
            \PY{k}{return} \PY{n}{df2}
        
        \PY{n}{nfreqs}\PY{o}{=}\PY{n}{normalizeall}\PY{p}{(}\PY{n}{np}\PY{o}{.}\PY{n}{array}\PY{p}{(}\PY{n}{freqs}\PY{p}{)}\PY{p}{)}
        \PY{n}{nsubset\PYZus{}freqs}\PY{o}{=}\PY{n}{normalizeall}\PY{p}{(}\PY{n}{np}\PY{o}{.}\PY{n}{array}\PY{p}{(}\PY{n}{subset\PYZus{}freqs}\PY{p}{)}\PY{p}{)}
\end{Verbatim}


    \subsection{The Algorithm}\label{the-algorithm}

\subsubsection{Distance measures: Points and
Clusters}\label{distance-measures-points-and-clusters}

Need: Distance between points (Euclidean for now) outout= matrix,
Distance between Clusters (average distance between points in each
cluster)

The algorithm requires several functions, one of which is a function to
find the distance between two points of arbitrary dimensionality. My
Function Point Distance calculates Euclidean distance, instead of L1,
because of \%these reasons which sound v important\%.

My next function is to find the center of the group of points. It simply
takes the average of the values in all dimensions.

    \begin{Verbatim}[commandchars=\\\{\}]
{\color{incolor}In [{\color{incolor}3}]:} \PY{k}{def} \PY{n+nf}{pointDistance} \PY{p}{(}\PY{n}{point1}\PY{p}{,} \PY{n}{point2}\PY{p}{)}\PY{p}{:}
            \PY{n}{dimNum}\PY{o}{=}\PY{n+nb}{len}\PY{p}{(}\PY{n}{point1}\PY{p}{)}
            \PY{k}{if} \PY{n}{dimNum}\PY{o}{!=}\PY{n+nb}{len}\PY{p}{(}\PY{n}{point2}\PY{p}{)}\PY{p}{:}
                \PY{n+nb}{print}\PY{p}{(}\PY{l+s+s2}{\PYZdq{}}\PY{l+s+s2}{we have a dimensionality problem, Houston}\PY{l+s+s2}{\PYZdq{}}\PY{p}{)}
                \PY{n+nb}{print}\PY{p}{(}\PY{n}{dimNum}\PY{p}{,}\PY{n+nb}{len}\PY{p}{(}\PY{n}{point2}\PY{p}{)}\PY{p}{)}       
        
            \PY{n}{dif}\PY{o}{=}\PY{n}{point1}\PY{p}{[}\PY{p}{:}\PY{o}{\PYZhy{}}\PY{l+m+mi}{1}\PY{p}{]}\PY{o}{\PYZhy{}}\PY{n}{point2}\PY{p}{[}\PY{p}{:}\PY{o}{\PYZhy{}}\PY{l+m+mi}{1}\PY{p}{]}
            \PY{n}{s}\PY{o}{=}\PY{n+nb}{sum}\PY{p}{(}\PY{n}{dif}\PY{o}{.}\PY{n}{T}\PY{o}{*}\PY{n}{dif}\PY{p}{)}
            \PY{n}{ssqrt}\PY{o}{=}\PY{n}{np}\PY{o}{.}\PY{n}{sqrt}\PY{p}{(}\PY{n}{s}\PY{p}{)}
            \PY{c+c1}{\PYZsh{}print(\PYZsq{}qrt2\PYZsq{},ssqrt)}
            \PY{k}{return} \PY{n}{ssqrt}
         
        \PY{k}{def} \PY{n+nf}{findCenter}\PY{p}{(}\PY{n}{points}\PY{p}{)}\PY{p}{:}
            \PY{n}{n}\PY{o}{=}\PY{n}{np}\PY{o}{.}\PY{n}{shape}\PY{p}{(}\PY{n}{points}\PY{p}{)}\PY{p}{[}\PY{l+m+mi}{0}\PY{p}{]}
            \PY{n}{d}\PY{o}{=}\PY{n}{np}\PY{o}{.}\PY{n}{shape}\PY{p}{(}\PY{n}{points}\PY{p}{)}\PY{p}{[}\PY{l+m+mi}{1}\PY{p}{]}
            \PY{n}{center}\PY{o}{=}\PY{n}{np}\PY{o}{.}\PY{n}{zeros}\PY{p}{(}\PY{n}{d}\PY{p}{)}
            \PY{k}{for} \PY{n}{i} \PY{o+ow}{in} \PY{n+nb}{range}\PY{p}{(}\PY{l+m+mi}{0}\PY{p}{,}\PY{n}{d}\PY{p}{)}\PY{p}{:}
                \PY{n}{center}\PY{p}{[}\PY{n}{i}\PY{p}{]}\PY{o}{=} \PY{n}{np}\PY{o}{.}\PY{n}{mean}\PY{p}{(}\PY{n}{points}\PY{o}{.}\PY{n}{iloc}\PY{p}{[}\PY{p}{:}\PY{p}{,}\PY{n}{i}\PY{p}{]}\PY{p}{)}
            \PY{k}{return} \PY{n}{center}
\end{Verbatim}


    \subsubsection{Distance Matrix}\label{distance-matrix}

Then the remaining steps are to calculate the distance between all
groups and merge the closest groups. To calculate the distance between
groups, I'm using the distance between the averages, instead of the
average distance between points, for the sake of efficiency.

    \begin{Verbatim}[commandchars=\\\{\}]
{\color{incolor}In [{\color{incolor}4}]:} \PY{n}{data}\PY{o}{=}\PY{n}{nsubset\PYZus{}freqs}\PY{o}{.}\PY{n}{copy}\PY{p}{(}\PY{p}{)}
        \PY{n}{n}\PY{o}{=}\PY{n}{np}\PY{o}{.}\PY{n}{shape}\PY{p}{(}\PY{n}{data}\PY{p}{)}\PY{p}{[}\PY{l+m+mi}{0}\PY{p}{]}
        \PY{n}{d}\PY{o}{=}\PY{n}{np}\PY{o}{.}\PY{n}{shape}\PY{p}{(}\PY{n}{data}\PY{p}{)}\PY{p}{[}\PY{l+m+mi}{1}\PY{p}{]}
        
        \PY{n}{df}\PY{o}{=}\PY{n}{pd}\PY{o}{.}\PY{n}{DataFrame}\PY{p}{(}\PY{n}{data}\PY{p}{)}
        \PY{n}{df}\PY{p}{[}\PY{l+s+s1}{\PYZsq{}}\PY{l+s+s1}{cluster}\PY{l+s+s1}{\PYZsq{}}\PY{p}{]}\PY{o}{=}\PY{n}{df}\PY{o}{.}\PY{n}{index}
        \PY{n}{centers}\PY{o}{=}\PY{n}{df}
        
        \PY{n}{distanceMatrix}\PY{o}{=}\PY{n}{np}\PY{o}{.}\PY{n}{zeros}\PY{p}{(}\PY{p}{(}\PY{n}{n}\PY{p}{,}\PY{n}{n}\PY{p}{)}\PY{p}{)}
        
        \PY{k}{for} \PY{n}{i} \PY{o+ow}{in} \PY{n+nb}{range}\PY{p}{(}\PY{l+m+mi}{0}\PY{p}{,}\PY{n}{n}\PY{p}{)}\PY{p}{:}
            \PY{n}{pointa} \PY{o}{=} \PY{n}{centers}\PY{o}{.}\PY{n}{iloc}\PY{p}{[}\PY{n}{i}\PY{p}{,}\PY{p}{]}
            \PY{k}{for} \PY{n}{j} \PY{o+ow}{in} \PY{n+nb}{range}\PY{p}{(}\PY{l+m+mi}{0}\PY{p}{,}\PY{n}{n}\PY{p}{)}\PY{p}{:}
                \PY{k}{if} \PY{n}{j}\PY{o}{\PYZlt{}}\PY{o}{=}\PY{n}{i}\PY{p}{:}
                    \PY{n}{distanceMatrix}\PY{p}{[}\PY{n}{i}\PY{p}{]}\PY{p}{[}\PY{n}{j}\PY{p}{]}\PY{o}{=}\PY{k+kc}{None}
                \PY{k}{else}\PY{p}{:}
                    \PY{n}{pointb} \PY{o}{=} \PY{n}{centers}\PY{o}{.}\PY{n}{iloc}\PY{p}{[}\PY{n}{j}\PY{p}{,}\PY{p}{]} 
                    \PY{n}{distanceMatrix}\PY{p}{[}\PY{n}{i}\PY{p}{]}\PY{p}{[}\PY{n}{j}\PY{p}{]}\PY{o}{=}\PY{n}{pointDistance}\PY{p}{(}\PY{n}{pointa}\PY{p}{,} \PY{n}{pointb}\PY{p}{)}
                    \PY{c+c1}{\PYZsh{}print(i,j,distanceMatrix[i][j])}
        
        \PY{n}{matrixCopy}\PY{o}{=}\PY{n}{distanceMatrix}\PY{o}{.}\PY{n}{copy}\PY{p}{(}\PY{p}{)}            
        \PY{n}{history}\PY{o}{=}\PY{n}{pd}\PY{o}{.}\PY{n}{DataFrame}\PY{p}{(}\PY{p}{\PYZob{}}\PY{l+s+s1}{\PYZsq{}}\PY{l+s+s1}{round1}\PY{l+s+s1}{\PYZsq{}}\PY{p}{:} \PY{n}{df}\PY{p}{[}\PY{l+s+s1}{\PYZsq{}}\PY{l+s+s1}{cluster}\PY{l+s+s1}{\PYZsq{}}\PY{p}{]}\PY{p}{\PYZcb{}}\PY{p}{)}
\end{Verbatim}


    \subsubsection{Merging Groups and Recording
History}\label{merging-groups-and-recording-history}

To merge points, I'm simply changing the group label as represented by
the number to the smaller of the two numbers. We have to change the
label for all points in the larger group and then recalculate the center
given the center for the smaller group with the new points.

The process of finding where certain values equal a given number, using
the packages that I'm using (specifically, numpy and pandas), is a bit
arcane. I create a boolean mask array which has true values wherever the
original array equals the values I'm searching for, and then performs
certain actions in the original array indexed by the mask. I'm storing
the current state of the cluster assignments in an array/matrix called
History so I can see how the hierarchical information in the tree came
together. I'm also tracking this in a two dimensional array that lists
which groups have merged at each step.

    \begin{Verbatim}[commandchars=\\\{\}]
{\color{incolor}In [{\color{incolor}5}]:} \PY{c+c1}{\PYZsh{}distanceMatrix=matrixCopy.copy()}
        \PY{c+c1}{\PYZsh{}data=nsubset\PYZus{}freqs.copy()}
        \PY{n}{df}\PY{o}{=}\PY{n}{pd}\PY{o}{.}\PY{n}{DataFrame}\PY{p}{(}\PY{n}{data}\PY{p}{)}
        \PY{n}{df}\PY{p}{[}\PY{l+s+s1}{\PYZsq{}}\PY{l+s+s1}{cluster}\PY{l+s+s1}{\PYZsq{}}\PY{p}{]}\PY{o}{=}\PY{n}{df}\PY{o}{.}\PY{n}{index}
        \PY{n}{centers}\PY{o}{=}\PY{n}{df}\PY{o}{.}\PY{n}{copy}\PY{p}{(}\PY{p}{)}
        \PY{n}{merges}\PY{o}{=}\PY{n}{pd}\PY{o}{.}\PY{n}{DataFrame}\PY{p}{(}\PY{n}{np}\PY{o}{.}\PY{n}{zeros}\PY{p}{(}\PY{p}{(}\PY{n}{n}\PY{p}{,}\PY{l+m+mi}{3}\PY{p}{)}\PY{p}{)}\PY{p}{,}\PY{n}{columns}\PY{o}{=}\PY{p}{[}\PY{l+s+s1}{\PYZsq{}}\PY{l+s+s1}{head}\PY{l+s+s1}{\PYZsq{}}\PY{p}{,} \PY{l+s+s1}{\PYZsq{}}\PY{l+s+s1}{leaves}\PY{l+s+s1}{\PYZsq{}}\PY{p}{,} \PY{l+s+s1}{\PYZsq{}}\PY{l+s+s1}{groupsize}\PY{l+s+s1}{\PYZsq{}}\PY{p}{]}\PY{p}{)}
        
        \PY{n}{iternum}\PY{o}{=}\PY{l+m+mi}{0}
        \PY{k}{while} \PY{n}{np}\PY{o}{.}\PY{n}{nanmin}\PY{p}{(}\PY{n}{distanceMatrix}\PY{p}{)}\PY{o}{\PYZgt{}}\PY{l+m+mi}{0}\PY{p}{:}
            \PY{n}{minimum}\PY{o}{=}\PY{n}{np}\PY{o}{.}\PY{n}{nanmin}\PY{p}{(}\PY{n}{distanceMatrix}\PY{p}{)}
            
        \PY{c+c1}{\PYZsh{} get indexes where distance is minimum. find smaller group}
            \PY{n}{mask}\PY{o}{=}\PY{n}{np}\PY{o}{.}\PY{n}{isin}\PY{p}{(}\PY{n}{distanceMatrix}\PY{p}{,} \PY{n}{minimum}\PY{p}{)}
            \PY{n}{a}\PY{p}{,}\PY{n}{b} \PY{o}{=}\PY{n}{np}\PY{o}{.}\PY{n}{where}\PY{p}{(}\PY{n}{mask}\PY{p}{)}
            \PY{n}{groupa}\PY{o}{=}\PY{n+nb}{min}\PY{p}{(}\PY{n}{a}\PY{p}{[}\PY{l+m+mi}{0}\PY{p}{]}\PY{p}{,}\PY{n}{b}\PY{p}{[}\PY{l+m+mi}{0}\PY{p}{]}\PY{p}{)}
            \PY{n}{groupb}\PY{o}{=}\PY{n+nb}{max}\PY{p}{(}\PY{n}{a}\PY{p}{[}\PY{l+m+mi}{0}\PY{p}{]}\PY{p}{,}\PY{n}{b}\PY{p}{[}\PY{l+m+mi}{0}\PY{p}{]}\PY{p}{)}
        
        \PY{c+c1}{\PYZsh{}    print(\PYZsq{}groupb\PYZsq{}, groupb, \PYZsq{}cluster\PYZsq{}, df.loc[groupb,\PYZsq{}cluster\PYZsq{}], \PYZsq{}groupa\PYZsq{}, groupa)}
            
        \PY{c+c1}{\PYZsh{}Find indexes of points in old group.}
        
            \PY{n}{mask}\PY{o}{=}\PY{n}{np}\PY{o}{.}\PY{n}{isin}\PY{p}{(}\PY{n}{df}\PY{p}{[}\PY{l+s+s1}{\PYZsq{}}\PY{l+s+s1}{cluster}\PY{l+s+s1}{\PYZsq{}}\PY{p}{]}\PY{p}{,} \PY{n}{groupb} \PY{p}{)}
            \PY{n}{index}\PY{o}{=}\PY{n}{np}\PY{o}{.}\PY{n}{where}\PY{p}{(}\PY{n}{mask}\PY{p}{)}
        \PY{c+c1}{\PYZsh{}    print(index)}
        \PY{c+c1}{\PYZsh{}    print(\PYZsq{}455\PYZsq{},df.loc[455,\PYZsq{}cluster\PYZsq{}], \PYZsq{}85\PYZsq{}, df.loc[85,\PYZsq{}cluster\PYZsq{},],\PYZsq{}48\PYZsq{},df.loc[48,\PYZsq{}cluster\PYZsq{}])}
            
            \PY{n}{df}\PY{o}{.}\PY{n}{loc}\PY{p}{[}\PY{n}{mask}\PY{p}{,}\PY{l+s+s1}{\PYZsq{}}\PY{l+s+s1}{cluster}\PY{l+s+s1}{\PYZsq{}}\PY{p}{]}\PY{o}{=}\PY{n+nb}{int}\PY{p}{(}\PY{n}{df}\PY{o}{.}\PY{n}{loc}\PY{p}{[}\PY{n}{groupa}\PY{p}{,}\PY{l+s+s1}{\PYZsq{}}\PY{l+s+s1}{cluster}\PY{l+s+s1}{\PYZsq{}}\PY{p}{]}\PY{p}{)}
            
            
        \PY{c+c1}{\PYZsh{}    print(\PYZsq{}groupb\PYZsq{}, groupb, \PYZsq{}cluster\PYZsq{}, df.loc[groupb,\PYZsq{}cluster\PYZsq{}], \PYZsq{}groupa\PYZsq{}, groupa)}
        
            \PY{n}{mask}\PY{o}{=}\PY{n}{np}\PY{o}{.}\PY{n}{isin}\PY{p}{(}\PY{n}{df}\PY{p}{[}\PY{l+s+s1}{\PYZsq{}}\PY{l+s+s1}{cluster}\PY{l+s+s1}{\PYZsq{}}\PY{p}{]}\PY{p}{,} \PY{n}{groupa}\PY{p}{)}
            \PY{c+c1}{\PYZsh{}print(mask)}
        
            \PY{n}{groupPoints}\PY{o}{=} \PY{n}{df}\PY{p}{[}\PY{n}{mask}\PY{p}{]}
        
        \PY{c+c1}{\PYZsh{}    print(iternum, \PYZsq{}group\PYZsq{}, groupa,\PYZsq{}n point(s) in group\PYZsq{}, len(groupPoints))}
            \PY{n}{centers}\PY{o}{.}\PY{n}{iloc}\PY{p}{[}\PY{n}{groupa}\PY{p}{]}\PY{o}{=}\PY{n}{findCenter}\PY{p}{(}\PY{n}{groupPoints}\PY{p}{)}
            \PY{n}{centers}\PY{o}{.}\PY{n}{iloc}\PY{p}{[}\PY{n}{groupb}\PY{p}{]}\PY{o}{=}\PY{k+kc}{None}
        
            \PY{n}{merges}\PY{o}{.}\PY{n}{loc}\PY{p}{[}\PY{n}{iternum}\PY{p}{,}\PY{l+s+s1}{\PYZsq{}}\PY{l+s+s1}{head}\PY{l+s+s1}{\PYZsq{}}\PY{p}{]}\PY{o}{=}\PY{n}{groupa}
            \PY{n}{merges}\PY{o}{.}\PY{n}{loc}\PY{p}{[}\PY{n}{iternum}\PY{p}{,}\PY{l+s+s1}{\PYZsq{}}\PY{l+s+s1}{leaves}\PY{l+s+s1}{\PYZsq{}}\PY{p}{]}\PY{o}{=}\PY{n}{groupb}
            \PY{n}{merges}\PY{o}{.}\PY{n}{loc}\PY{p}{[}\PY{n}{iternum}\PY{p}{,}\PY{l+s+s1}{\PYZsq{}}\PY{l+s+s1}{groupsize}\PY{l+s+s1}{\PYZsq{}}\PY{p}{]}\PY{o}{=}\PY{n+nb}{len}\PY{p}{(}\PY{n}{groupPoints}\PY{p}{)}
        
        \PY{c+c1}{\PYZsh{}    print(\PYZsq{}new center\PYZsq{},centers.iloc[groupa])}
        
            \PY{n}{history}\PY{p}{[}\PY{n+nb}{str}\PY{p}{(}\PY{n}{iternum}\PY{p}{)}\PY{p}{]}\PY{o}{=}\PY{n}{df}\PY{p}{[}\PY{l+s+s1}{\PYZsq{}}\PY{l+s+s1}{cluster}\PY{l+s+s1}{\PYZsq{}}\PY{p}{]}
            \PY{n}{distanceMatrix}\PY{p}{[}\PY{n}{groupb}\PY{p}{,}\PY{p}{:}\PY{p}{]}\PY{o}{=}\PY{k+kc}{None}
            \PY{n}{distanceMatrix}\PY{p}{[}\PY{p}{:}\PY{p}{,}\PY{n}{groupb}\PY{p}{]}\PY{o}{=}\PY{k+kc}{None}
            
                  
            \PY{n}{i}\PY{o}{=}\PY{n}{groupa}
            \PY{n}{pointa} \PY{o}{=} \PY{n}{centers}\PY{o}{.}\PY{n}{iloc}\PY{p}{[}\PY{n}{i}\PY{p}{]}
            \PY{k}{for} \PY{n}{j} \PY{o+ow}{in} \PY{n+nb}{range}\PY{p}{(}\PY{l+m+mi}{0}\PY{p}{,}\PY{n}{n}\PY{p}{)}\PY{p}{:}
                \PY{k}{if} \PY{n}{j}\PY{o}{\PYZgt{}}\PY{n}{groupa}\PY{p}{:}
                    \PY{n}{pointb} \PY{o}{=} \PY{n}{centers}\PY{o}{.}\PY{n}{iloc}\PY{p}{[}\PY{n}{j}\PY{p}{,}\PY{p}{]} 
                    \PY{n}{distanceMatrix}\PY{p}{[}\PY{n}{i}\PY{p}{]}\PY{p}{[}\PY{n}{j}\PY{p}{]}\PY{o}{=}\PY{n}{pointDistance}\PY{p}{(}\PY{n}{pointa}\PY{p}{,} \PY{n}{pointb}\PY{p}{)}
                    \PY{c+c1}{\PYZsh{}print(\PYZdq{}changed a to b to value\PYZdq{}, pointa, pointb)}
            \PY{n}{iternum}\PY{o}{+}\PY{o}{=}\PY{l+m+mi}{1}
            
        
        \PY{c+c1}{\PYZsh{}\PYZsh{} recalculate distances between c and all other centers. }
\end{Verbatim}


    \begin{Verbatim}[commandchars=\\\{\}]
/home/kchalk/anaconda3/lib/python3.6/site-packages/ipykernel\_launcher.py:9: RuntimeWarning: All-NaN slice encountered
  if \_\_name\_\_ == '\_\_main\_\_':

    \end{Verbatim}

    \begin{Verbatim}[commandchars=\\\{\}]
{\color{incolor}In [{\color{incolor}6}]:} \PY{n}{history}\PY{o}{.}\PY{n}{index}\PY{o}{=}\PY{n}{subset\PYZus{}inputs}\PY{p}{[}\PY{l+s+s1}{\PYZsq{}}\PY{l+s+s1}{subreddit}\PY{l+s+s1}{\PYZsq{}}\PY{p}{]}
        
        \PY{n}{merges}\PY{o}{.}\PY{n}{groupby}\PY{p}{(}\PY{l+s+s1}{\PYZsq{}}\PY{l+s+s1}{head}\PY{l+s+s1}{\PYZsq{}}\PY{p}{)}\PY{o}{.}\PY{n}{count}\PY{p}{(}\PY{p}{)}\PY{o}{.}\PY{n}{sort\PYZus{}values}\PY{p}{(}\PY{l+s+s1}{\PYZsq{}}\PY{l+s+s1}{leaves}\PY{l+s+s1}{\PYZsq{}}\PY{p}{,} \PY{n}{desc}\PY{o}{=}\PY{k+kc}{True}\PY{p}{)}
\end{Verbatim}


    \begin{Verbatim}[commandchars=\\\{\}]

        ---------------------------------------------------------------------------

        TypeError                                 Traceback (most recent call last)

        <ipython-input-6-b6901fb9cbb1> in <module>()
          1 history.index=subset\_inputs['subreddit']
          2 
    ----> 3 merges.groupby('head').count().sort\_values('leaves', desc=True)
    

        TypeError: sort\_values() got an unexpected keyword argument 'desc'

    \end{Verbatim}

    \section{Graphs}\label{graphs}

\subparagraph{what I have graphed}\label{what-i-have-graphed}

In this graph, I projected the data onto two dimensional PCA space. I am
graphing all 600 points in a scatter plot colored by cluster. Because of
the size of the dataset, the only thing that one can actually see is
that the number of colors reduces as the algorithm runs longer.

    \begin{Verbatim}[commandchars=\\\{\}]
{\color{incolor}In [{\color{incolor} }]:} \PY{k+kn}{from} \PY{n+nn}{sklearn}\PY{n+nn}{.}\PY{n+nn}{decomposition} \PY{k}{import} \PY{n}{PCA}
        
        \PY{n}{pca}\PY{o}{=}\PY{n}{PCA}\PY{p}{(}\PY{n}{n\PYZus{}components}\PY{o}{=}\PY{l+m+mi}{11}\PY{p}{)}
        \PY{n}{pcs}\PY{o}{=}\PY{n}{pca}\PY{o}{.}\PY{n}{fit\PYZus{}transform}\PY{p}{(}\PY{n}{subset\PYZus{}freqs}\PY{p}{)}
        \PY{n}{plotData}\PY{o}{=}\PY{n}{pd}\PY{o}{.}\PY{n}{DataFrame}\PY{p}{(}\PY{n}{pcs}\PY{p}{,} \PY{n}{index}\PY{o}{=}\PY{n}{subset\PYZus{}inputs}\PY{p}{[}\PY{l+s+s1}{\PYZsq{}}\PY{l+s+s1}{subreddit}\PY{l+s+s1}{\PYZsq{}}\PY{p}{]}\PY{p}{)}
        
        \PY{n}{plots}\PY{o}{=}\PY{p}{[}\PY{p}{]}
        \PY{n}{j}\PY{o}{=}\PY{l+m+mi}{0}
        
        \PY{n}{fig} \PY{o}{=} \PY{n}{plt}\PY{o}{.}\PY{n}{figure}\PY{p}{(}\PY{n}{figsize}\PY{o}{=}\PY{p}{(}\PY{l+m+mi}{14}\PY{p}{,}\PY{l+m+mi}{28}\PY{p}{)}\PY{p}{)}
        
        
        \PY{k}{for} \PY{n}{i} \PY{o+ow}{in} \PY{n+nb}{range}\PY{p}{(}\PY{l+m+mi}{0}\PY{p}{,}\PY{l+m+mi}{601}\PY{p}{,} \PY{l+m+mi}{50}\PY{p}{)}\PY{p}{:}
            \PY{n}{plotData}\PY{p}{[}\PY{l+s+s1}{\PYZsq{}}\PY{l+s+s1}{cluster}\PY{l+s+s1}{\PYZsq{}}\PY{p}{]}\PY{o}{=}\PY{n}{history}\PY{o}{.}\PY{n}{loc}\PY{p}{[}\PY{p}{:}\PY{p}{,}\PY{n+nb}{str}\PY{p}{(}\PY{n}{i}\PY{p}{)}\PY{p}{]}
            \PY{n}{j}\PY{o}{+}\PY{o}{=}\PY{l+m+mi}{1}
            \PY{n}{fig}\PY{o}{.}\PY{n}{add\PYZus{}subplot}\PY{p}{(}\PY{l+m+mi}{7}\PY{p}{,} \PY{l+m+mi}{2}\PY{p}{,} \PY{n}{j}\PY{p}{)}
            \PY{n}{plt}\PY{o}{.}\PY{n}{scatter}\PY{p}{(}\PY{n}{x}\PY{o}{=}\PY{n}{plotData}\PY{p}{[}\PY{l+m+mi}{0}\PY{p}{]}\PY{p}{,}\PY{n}{y}\PY{o}{=}\PY{n}{plotData}\PY{p}{[}\PY{l+m+mi}{1}\PY{p}{]}\PY{p}{,} \PY{n}{c}\PY{o}{=}\PY{n}{plotData}\PY{p}{[}\PY{l+s+s1}{\PYZsq{}}\PY{l+s+s1}{cluster}\PY{l+s+s1}{\PYZsq{}}\PY{p}{]}\PY{p}{)}
        \PY{n}{plt}\PY{o}{.}\PY{n}{show}\PY{p}{(}\PY{p}{)}
\end{Verbatim}


    \subparagraph{what I should graph}\label{what-i-should-graph}

I would like to have drawn trees and subtrees for showing the merge
history of the clusters, but I experienced difficulty with graphing
trees in python.

Another graph that I could make is to show the points in a given group
as time moves on, which would show at what point large clusters merge
together.

    \begin{Verbatim}[commandchars=\\\{\}]
{\color{incolor}In [{\color{incolor} }]:} \PY{n}{fig} \PY{o}{=} \PY{n}{plt}\PY{o}{.}\PY{n}{figure}\PY{p}{(}\PY{n}{figsize}\PY{o}{=}\PY{p}{(}\PY{l+m+mi}{14}\PY{p}{,}\PY{l+m+mi}{28}\PY{p}{)}\PY{p}{)}
        
        \PY{n}{j}\PY{o}{=}\PY{l+m+mi}{0}
        \PY{k}{for} \PY{n}{i} \PY{o+ow}{in} \PY{n+nb}{range}\PY{p}{(}\PY{l+m+mi}{0}\PY{p}{,}\PY{l+m+mi}{299}\PY{p}{,} \PY{l+m+mi}{100}\PY{p}{)}\PY{p}{:}
            \PY{n}{plotData}\PY{o}{=}\PY{n}{pd}\PY{o}{.}\PY{n}{DataFrame}\PY{p}{(}\PY{n}{pcs}\PY{p}{,} \PY{n}{index}\PY{o}{=}\PY{n}{subset\PYZus{}inputs}\PY{p}{[}\PY{l+s+s1}{\PYZsq{}}\PY{l+s+s1}{subreddit}\PY{l+s+s1}{\PYZsq{}}\PY{p}{]}\PY{p}{)}
            \PY{n}{plotData}\PY{p}{[}\PY{l+s+s1}{\PYZsq{}}\PY{l+s+s1}{cluster}\PY{l+s+s1}{\PYZsq{}}\PY{p}{]}\PY{o}{=}\PY{n}{history}\PY{o}{.}\PY{n}{loc}\PY{p}{[}\PY{p}{:}\PY{p}{,}\PY{n+nb}{str}\PY{p}{(}\PY{n}{i}\PY{p}{)}\PY{p}{]}
            \PY{n}{mask}\PY{o}{=}\PY{n}{np}\PY{o}{.}\PY{n}{isin}\PY{p}{(}\PY{n}{plotData}\PY{p}{[}\PY{l+s+s1}{\PYZsq{}}\PY{l+s+s1}{cluster}\PY{l+s+s1}{\PYZsq{}}\PY{p}{]}\PY{p}{,} \PY{p}{[}\PY{l+m+mi}{12}\PY{p}{,}\PY{l+m+mi}{0}\PY{p}{,}\PY{l+m+mi}{5}\PY{p}{,}\PY{l+m+mi}{16}\PY{p}{,}\PY{l+m+mi}{4}\PY{p}{,}\PY{l+m+mi}{140}\PY{p}{,}\PY{l+m+mi}{1}\PY{p}{,}\PY{l+m+mi}{26}\PY{p}{,}\PY{l+m+mi}{111}\PY{p}{]}\PY{p}{)}
            \PY{n}{plotData}\PY{o}{=}\PY{n}{plotData}\PY{o}{.}\PY{n}{loc}\PY{p}{[}\PY{n}{mask}\PY{p}{]}
            \PY{n}{j}\PY{o}{+}\PY{o}{=}\PY{l+m+mi}{1}
            \PY{n}{fig}\PY{o}{.}\PY{n}{add\PYZus{}subplot}\PY{p}{(}\PY{l+m+mi}{9}\PY{p}{,} \PY{l+m+mi}{3}\PY{p}{,} \PY{n}{j}\PY{p}{)}
            \PY{n}{plt}\PY{o}{.}\PY{n}{scatter}\PY{p}{(}\PY{n}{x}\PY{o}{=}\PY{n}{plotData}\PY{p}{[}\PY{l+m+mi}{0}\PY{p}{]}\PY{p}{,}\PY{n}{y}\PY{o}{=}\PY{n}{plotData}\PY{p}{[}\PY{l+m+mi}{1}\PY{p}{]}\PY{p}{,} \PY{n}{c}\PY{o}{=}\PY{n}{plotData}\PY{p}{[}\PY{l+s+s1}{\PYZsq{}}\PY{l+s+s1}{cluster}\PY{l+s+s1}{\PYZsq{}}\PY{p}{]}\PY{p}{,} \PY{n}{cmap}\PY{o}{=}\PY{l+s+s1}{\PYZsq{}}\PY{l+s+s1}{Paired}\PY{l+s+s1}{\PYZsq{}}\PY{p}{)}
        
        \PY{k}{for} \PY{n}{i} \PY{o+ow}{in} \PY{n+nb}{range}\PY{p}{(}\PY{l+m+mi}{300}\PY{p}{,}\PY{l+m+mi}{639}\PY{p}{,} \PY{l+m+mi}{20}\PY{p}{)}\PY{p}{:}
            \PY{n}{plotData}\PY{o}{=}\PY{n}{pd}\PY{o}{.}\PY{n}{DataFrame}\PY{p}{(}\PY{n}{pcs}\PY{p}{,} \PY{n}{index}\PY{o}{=}\PY{n}{subset\PYZus{}inputs}\PY{p}{[}\PY{l+s+s1}{\PYZsq{}}\PY{l+s+s1}{subreddit}\PY{l+s+s1}{\PYZsq{}}\PY{p}{]}\PY{p}{)}
            \PY{n}{plotData}\PY{p}{[}\PY{l+s+s1}{\PYZsq{}}\PY{l+s+s1}{cluster}\PY{l+s+s1}{\PYZsq{}}\PY{p}{]}\PY{o}{=}\PY{n}{history}\PY{o}{.}\PY{n}{loc}\PY{p}{[}\PY{p}{:}\PY{p}{,}\PY{n+nb}{str}\PY{p}{(}\PY{n}{i}\PY{p}{)}\PY{p}{]}
            \PY{n}{mask}\PY{o}{=}\PY{n}{np}\PY{o}{.}\PY{n}{isin}\PY{p}{(}\PY{n}{plotData}\PY{p}{[}\PY{l+s+s1}{\PYZsq{}}\PY{l+s+s1}{cluster}\PY{l+s+s1}{\PYZsq{}}\PY{p}{]}\PY{p}{,} \PY{p}{[}\PY{l+m+mi}{12}\PY{p}{,}\PY{l+m+mi}{0}\PY{p}{,}\PY{l+m+mi}{5}\PY{p}{,}\PY{l+m+mi}{16}\PY{p}{,}\PY{l+m+mi}{4}\PY{p}{,}\PY{l+m+mi}{140}\PY{p}{,}\PY{l+m+mi}{1}\PY{p}{,}\PY{l+m+mi}{26}\PY{p}{,}\PY{l+m+mi}{111}\PY{p}{]}\PY{p}{)}
            \PY{n}{plotData}\PY{o}{=}\PY{n}{plotData}\PY{o}{.}\PY{n}{loc}\PY{p}{[}\PY{n}{mask}\PY{p}{]}
            \PY{n}{j}\PY{o}{+}\PY{o}{=}\PY{l+m+mi}{1}
            \PY{n}{fig}\PY{o}{.}\PY{n}{add\PYZus{}subplot}\PY{p}{(}\PY{l+m+mi}{9}\PY{p}{,} \PY{l+m+mi}{3}\PY{p}{,} \PY{n}{j}\PY{p}{)}
            \PY{n}{plt}\PY{o}{.}\PY{n}{scatter}\PY{p}{(}\PY{n}{x}\PY{o}{=}\PY{n}{plotData}\PY{p}{[}\PY{l+m+mi}{0}\PY{p}{]}\PY{p}{,}\PY{n}{y}\PY{o}{=}\PY{n}{plotData}\PY{p}{[}\PY{l+m+mi}{1}\PY{p}{]}\PY{p}{,} \PY{n}{c}\PY{o}{=}\PY{n}{plotData}\PY{p}{[}\PY{l+s+s1}{\PYZsq{}}\PY{l+s+s1}{cluster}\PY{l+s+s1}{\PYZsq{}}\PY{p}{]}\PY{p}{,} \PY{n}{cmap}\PY{o}{=}\PY{l+s+s1}{\PYZsq{}}\PY{l+s+s1}{Paired}\PY{l+s+s1}{\PYZsq{}}\PY{p}{)}
        
        \PY{n}{plt}\PY{o}{.}\PY{n}{show}\PY{p}{(}\PY{p}{)}
\end{Verbatim}


    \section{Future Directions}\label{future-directions}

More data. More graphs. Point to point distance


    % Add a bibliography block to the postdoc
    
    
    
    \end{document}
